\documentclass[10pt]{extarticle}
\usepackage[margin=0.2in]{geometry}
\usepackage{romannum}
\usepackage[most]{tcolorbox}
\usepackage{enumitem}
\usepackage{hyperref}
\usepackage{tabularx}
\usepackage{multicol}
\usepackage{multirow}
\usepackage{fontawesome5}
\usepackage{tabularx, colortbl}
\setlist[itemize]{noitemsep, topsep=0pt}
\addtolength{\parskip}{-1.5mm}
\tcbset{
    frame code={}
    center title,
    left=0pt,
    right=0pt,
    top=0pt,
    bottom=0pt,
    colback=gray!40,
    colframe=white,
    width=\dimexpr\textwidth\relax,
    enlarge left by=0mm,
    boxsep=3pt,
    arc=0pt,outer arc=0pt,
    }
\begin{document}
\begin{flushleft}
\noindent {\huge\textbf{Ankit Lakhiwal}}
 \vspace{-2mm}
\end{flushleft}
Dual Degree\\
Department of Aerospace Engineering\hfill \faEnvelope\
\href{mailto:ankitl@iitk.ac.in}{ankitl@iitk.ac.in} $|$ \faPhone\ +91-9079661628
 \vspace{-1.5mm}
$|$ \faGithub\  \href{https://github.com/lakhiwal007}{lakhi007}\ \ \ \ \ 
\vspace{-2mm}
\\

\noindent\rule[0.5ex]{\linewidth}{1pt}
\vspace{-8mm}
{\large \textbf{\begin{tcolorbox}\textsc{Academic Qualifications}\end{tcolorbox}}}
\begin{center}
\begin{tabular}{|p{2.5cm}|p{6.0cm}|p{7.5cm}|p{2.8cm}|}
\hline
\centering{\textbf{Year}} & \centering{\textbf{Degree/Certificate}} & \centering{\textbf{Institute}} & \centering \textbf{CPI/$\%$} \tabularnewline
\hline
% \centering{2022}-Present & \centering{M.Tech} & \centering{Indian Institute of Technology Kanpur} & \centering 8/10 \tabularnewline \hline 
\centering{2018-Present} & \centering{Dual Degree} & \centering{Indian Institute of Technology Kanpur} & \centering PG 8 / UG 6.31 \tabularnewline \hline 
\centering{2017} & \centering{Class \Romannum{12} (RBSE)} & \centering{Mehta Public School, Jaipur} & \centering 88.89$\%$ \tabularnewline
\hline
\centering{2015} & \centering{Class \Romannum{10} (RBSE)} & \centering{Mehta Public School, Jaipur} & \centering 94$\%$ \tabularnewline
\hline
\end{tabular}
\end{center}
\vspace{-4.5mm}

% M.Tech Thesis
{\large \textbf{\begin{tcolorbox}\textsc{M.Tech. Thesis}\end{tcolorbox}}}
\vspace{-1mm}
\hspace{-1.3mm}\textbf{Scaling of Peripheral Vortex Combuster} $|$ \textbf{Advisor}: Prof. Vaibhav Arghode
\hfill\hfill\textit{May'22 - Present} 
\begin{itemize}
\item Performed simulation on prototype model for different \textbf{configuration of injectors} to study the flow circulation inside combuster.
\item Studied different scaling techniques, and performed simulation for scaled model and compared the results with prototype model.
\end{itemize}
\vspace{-5mm}


{\large \textbf{\begin{tcolorbox}\textsc{Project Work}\end{tcolorbox}}}
\vspace{-1mm}

% Django Project : StudyBudd
\hspace{-1.5mm}\textbf{Blogging Website} $|$ (Self Project)
\hfill\hfill \textit{\href{https://github.com/lakhiwal007/Django-Projects/tree/main/LakhiLab}{\faGithub} May'21 - June'21} 
\begin{itemize}
\item Created a blogging website using \textbf{Django} framework and \textbf{PostgreSQL} for backend  and \textbf{HTML, CSS} for frontend development.
\item User can sign up, login and logout using email and password and add, update and delete his post only when logged in and can see other users posts, profiles and comment on their posts and also can filter posts based on topics.Recent activity shows in sidebar.
\end{itemize}
\vspace{2.5mm}

% Django Project : Pexels
\hspace{-1.5mm}\textbf{Photo Gallery Website} $|$ (Self Project)
\hfill\hfill\textit{\href{https://github.com/lakhiwal007/nextJsProject}{\faGithub} May'22 - June'22} 
\begin{itemize}
\item Created a photo gallery website using \textbf{NextJS} framework and \textbf{Tailwind CSS} for frontend and \textbf{Firebase} to handle backend.
\item User can sign up, login and logout using email and password or can use google authentication and upload photos, add caption, tags, location and see the other users photos and can filter photos based on tags and location.
\end{itemize}
\vspace{1.5mm}

% Satellite data analysis
\hspace{-1.5mm}\textbf{Satellite Data Analysis} $|$ (Self Project)
\hfill\hfill\textit{\href{https://jovian.ai/lakhiwalankit0002/satellite-data-analysis}{\faDiceFour}May'21 - June'21} 
\begin{itemize}
\item Searched for satellite data then cleaned it and used \textbf{numpy}, \textbf{pandas}, \textbf{matplotlib} and \textbf{plotly} python libraries to do data analysis.
\item Final results shows that which country have how many satellites for different works, number of satellites in different orbits, which company have largest number of satellite launched in space, from where are the most satellites are launched.
\end{itemize}
\vspace{1.5mm}

% covid 19 data analysis
\hspace{-1.5mm}\textbf{Covid-19 Data Analysis} $|$ (Self Project)
\hfill\hfill\textit{\href{https://public.tableau.com/app/profile/ankit.lakhiwal/viz/Covid-19DataAnalysis_16276722946220/Dashboard1}{\faChartPie} July'21} 
\begin{itemize}
% \item Analysed Covid-19 data using dataset from Kaggle. 
\item Searched for Covid-19 data on \textbf{Kaggle}, for graphs used \textbf{Tableau software}.Final graphical representation shows that how many recovered, active, confirmed and deaths cases, in a country, in WHO regions and in different quarters of 2020 year.
\end{itemize}
\vspace{1.5mm}

% fifa match data analysis
\hspace{-1.5mm}\textbf{FIFA Matches Data Analysis} $|$ (Self Project)
\hfill\hfill\textit{\href{https://public.tableau.com/app/profile/ankit.lakhiwal/viz/Fifamatchdataanalysis/Sheet6}{\faChartPie} Oct'21} 
\begin{itemize}
\item Searched for FIFA match data on \textbf{Kaggle}, for graphs used Tableau software.Final graphical representation shows that which country won and which one runner up most the FIFA cup and in which city and stadium the most of the matches were played.
\end{itemize}
\vspace{1.5mm}

% Ornithopter
\hspace{-2.7mm}
\textbf{Ornithopter} $|$ Summer Project: Aeromodelling Club, IIT Kanpur
\hfill\hfill\textit{\href{https://drive.google.com/drive/folders/10ncwIcoz3mCMsF_QOLuh80orXczK1xUd?usp=sharing}{\faGoogleDrive} April'19 - July'19} 
\begin{itemize}
\item Developed a bird mimicking flapping wing aircraft which generates lift by flapping its wings and controls direction using the tail.
\item Successfully created lightweight(24gms) and \textbf{fully remote controlled} aircraft after including battery, servo and ESC in final model.
\item Achieved the \textbf{best summer project} award under Aeromodelling club under science and technology council.
\end{itemize}
\vspace{2.5mm}

% 2D wire bending Machine
\hspace{-1.5mm}\textbf{2D Wire Bending Machine} $|$ Semester Project : TA202A, IIT Kanpur
\hfill\hfill\textit{\href{https://drive.google.com/drive/folders/1w-86oNq7zGUXlRwYd7UDS7DIbX-KlrYc?usp=sharing}{\faGoogleDrive} August'20 - November'20} 
\begin{itemize}
\item Created a wire bending machine which can create simple 2d shapes like Triangle, Square, Pentagon etc.
\item Created parts using \textbf{Autocad Inventor} and printed it using \textbf{3d printer} and write \textbf{Arduino} code to create different 2d shapes.
\end{itemize}
\vspace{2mm}
{\large \textbf{\begin{tcolorbox}\textsc{Position of Responsibility}\end{tcolorbox}}}
\vspace{-1mm}
\hspace{3.5mm}\textbf{Secretary} $|$ Aeromodelling Club, IIT Kanpur
\hfill\hfill\textit{July'19 - March'20}
\begin{itemize}
\item Responsible for the proper functioning of the club and mentored and fostered freshmen students in developing skills.
\end{itemize}
\begin{itemize}
\item Organised workshops to develop skills like radio control design and fabrication and quizzes for aeromodelling enthusiasts.
\end{itemize}

 \vspace{-5.0mm}
{\large \textbf{\begin{tcolorbox}\textsc{Social Work}\end{tcolorbox}}}
\vspace{-1mm}
\hspace{-3.0mm}
\textbf{Workshop Guide} $|$ Aeromodelling, IIT Kanpur, 2019
% \hfill\hfill\textit{July'19 - Present}
\begin{itemize}
\item Conducted introductory workshop for freshers to showcase the aeromodelling  club projects.
\item Guide freshers to make remote control aircraft from fabrication, electronics connections for remote control flying.
\end{itemize}
\vspace{2mm}
\hspace{3.5mm}\textbf{Student Guide} $|$ Open House, IIT Kanpur, 2020
% \hfill\hfill\textit{April'20 - June'20}
\begin{itemize}
\item Volunteered and represented aeromodelling club on Diamond Jubilee celebrations of IIT Kanpur for school students.
\item Showcase the works of Aeromodelling club to school students and inspired them to think about innovative ideas.
\end{itemize}
\vspace{2.5mm}
{\large \textbf{\begin{tcolorbox}\textsc{Relevant Coursework \hfill \textit{\footnotesize{* Online Courses}} \hspace{3mm} }\end{tcolorbox}}}
\vspace{-3mm}
\begin{left}
\begin{tabular}{l l l}
       Fundamentals of Computing (ESC101A) &
       Introduction to Numerical Methods (ME685A)&
       Image Processing (EE604A) \\
       Data structure and algorithm*&
       Data analysis with python* &
       Finite Element Method (AE635A)\\
       Compressible Aerodynamics (AE311A) &
       Flight Mechanics (AE321A) &
       Airbreathing Propulsion (AE341A) \\
       Aircraft Design (AE461 and AE462A) &
       Rocket Propulsion (AE441A) &
       Optimal Space Flight Control (AE777A) \\
       Space Dynamics(AE641A) &
       Experimental Flow Control (AE712A) &
       Boundary Layer Instability (AE617A) \\
\end{tabular}    
\end{left}
 \vspace{-2.5mm}
{\large \textbf{\begin{tcolorbox}\textsc{Technical Skills}\end{tcolorbox}}}
 \vspace{-2.5mm}
\begin{left}
\begin{tabular}{m{30em} | m{25em}}
       Programming Languages:  C++, Python, JavaScript.&  Web development: HTML, CSS, Bootstrap, Tailwind CSS \\ \hline
       Modeling Software:  Inventor, Fusion360, Ansys, MATLAB, XFLR5 & Frameworks: ReactJs, NextJs, Django \\ \hline
       Other: Linux, Tablue, MS Office, LabVIEW, VSCode &  Libraries: Numpy, Pandas, Matplotlib, Plotly\\
\end{tabular}    
\end{left}
\vspace{-2mm}
{\large \textbf{\begin{tcolorbox}\textsc{Extra-Curricular Activities}\end{tcolorbox}}}
\vspace{-3mm}
\begin{left}
\begin{tabular}{m{6em}| m{30em}}
\multirow{2}{*}{Technical} & \multicolumn{1}{l}{Participated in Boeing competition held in IIT Bombay in Techfest and achieved first position in first round.} \\
                                				 & \multicolumn{1}{l}{Participated in galaxy and represented hall under aeromodelling club. } 
                                                 \\\hline
   \multirow{2}{*}{Sports} & \multicolumn{1}{l}{Enthusiast in Hiking. Completed the Kedarkantha trek which is around 12000 ft in altitude.} \\
                                				 & \multicolumn{1}{l}{Fond of cycling. Usually rides in campus and sometimes to Ganga Bairaj in morning on weekends.}  \\
\end{tabular}
\end{left}
\end{document}
